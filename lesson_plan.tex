\documentclass[12pt]{article}
\usepackage{scrextend}
\usepackage[T2A]{fontenc}
\usepackage[utf8]{inputenc}
\usepackage[english,russian]{babel}
\usepackage{float}
\usepackage[pdftex]{graphicx}
\usepackage{microtype}
\usepackage{xcolor}
\usepackage{mdframed}
\usepackage[texcoord=true]{eso-pic}

\usepackage[%
	automark,
	headsepline,                %% Separation line below the header
	%  footsepline,               %% Separation line above the footer
	%markuppercase
]{scrpage2}

%\topmargin 0.5cm
%\leftmargin -1cm
%\textwidth 15cm
\textheight 21.5cm

\parindent 0.4in
%\parskip 0.7em


%Экспериментальная студия аудио-видео технологий

\usepackage{blindtext}

\AddToShipoutPictureBG{%
	\put(20,-120){\\
		\includegraphics[height=3cm]{zil_logo_color}\\
		%\color{lgray}\rule{\paperwidth}{7cm}
	}%
}

\lohead{
	%\includegraphics[width=1cm]{zil_logo_color}
	\hfill Экспериментальная студия аудио-видео технологий 
	\hfill \pagemark
	}    %% Top left on odd pages
\rohead{}
\chead{}  
\cfoot{}

\automark[subsection]{section}
\pagestyle{scrheadings}

\newcommand{\pdobj}[1]{
	\fbox{\texttt{#1}}
}

\mdfsetup{
	middlelinecolor	=red,
	middlelinewidth=2pt,
	backgroundcolor=gray!10,
	leftmargin=1cm,
	roundcorner=10pt,
	rightmargin=1cm
	}

\newcommand{\hometask}[1]{
	\begin{mdframed}
		{\textbf{Задание:}} \hspace{0.5cm} #1
	\end{mdframed}
}

\newcommand{\synth}{\texttt{synth1\~{}.pd}}

\newcommand{\synthhelp}{\texttt{synth1\~{}-help.pd}}

\begin{document}
	
\vspace*{0.5cm}

\noindent
{\huge Программа занятий}

\section{Вводное занятие}

\begin{enumerate}
	\item Установка PureData на компьютеры участников
	\item Обзор и знакомство с возможностями программы
	\item Подключение к микшерному пульту
	\item проверка звука \texttt{testtone.pd}
	\item <<Hello, world!>> и первые звуки
	\item звуковой оркестр и коллективная импровизация 
	\item объекты \pdobj{osc\~}, \pdobj{dac\~}, \pdobj{*\~}, \pdobj{hsl}, \pdobj{vsl}.
\end{enumerate}

\hometask{создание звукового интервала, аккорда. октава, квинта, трезвучие. }

\section{Постройка синтезатора}

\subsection{Первый монофонический синтезатор}
\begin{enumerate}
	\item инлеты -- аутлеты
	\item оформляем синтезатор для дальнейшего использования в виде патча \synth
	\item использование справки по объектам Pd.
	\item создание справки помощи \synthhelp
	\item комментирование патча для будущего использования
	\item объекты \pdobj{osc\~}, \pdobj{comment}, \pdobj{inlet}, \pdobj{inlet\~}, \pdobj{outlet}, \pdobj{outlet\~}.
\end{enumerate}

\hometask{использование нескольких \synth в одном патче, управление громкостью каждого. }

\subsection{Строим терменвокс}

\subsubsection{Управляемый мышью терменвокс}

\begin{enumerate}
	\item высота и амплитуда \synth \ управляется мышкой 
	\item Добавляем тремоло (амплитудная модуляция)
	\item Добавляем вибрацию (частотная модуляция)
	\item объекты \pdobj{cursor}, \pdobj{print}, \pdobj{route}, \pdobj{+\~}, \pdobj{+}, \pdobj{nbx}, \pdobj{msg}
\end{enumerate}

\hometask{сделать терменвокс, где параметр курсора \textbf{x} влияет на амплитуду, а \textbf{y} частоту. }

\hometask{сделать терменвокс, где параметр курсора \textbf{x} влияет на частоту, а \textbf{y} амплитуду вибрации. }

%\hometask{многоголосный синтезатор}

\subsubsection{Графический интерфейс к терменвоксу}
\begin{enumerate}
	\item использование сабпатчей. изменение цвета фона. обзор GUI элементов.
	\item объекты \pdobj{pd} \pdobj{cnv}, \pdobj{grid}, \pdobj{key}, \pdobj{keyup}, \pdobj{keyname} и все графические элементы
	\item добавляем индикаторы уровня сигнала
\end{enumerate}

\hometask{использовать все графические элементы для управления синтезатором}

\subsection{интерфейс MIDI}
\begin{enumerate}
	\item управляем синтезатором при помощи MIDI контролов
	\item подключаем MIDI клавиатуру
	\item объекты \pdobj{ctlin}, \pdobj{notein}, \pdobj{stripnote}
\end{enumerate}

\subsubsection{Мобильный телефон -- беспроводной контроллер терменвокса}

\begin{enumerate}
\item протокол OSC
\item настройка беспроводной сети
\item установка OSC клиента на телефон
\item соединение OSC и синтезатора на прошлом уроке
\item объекты \pdobj{import mrpeach}, \pdobj{udpreceive}, \pdobj{unpackOSC}, \pdobj{routeOSC}, \pdobj{==}  
\end{enumerate}

\subsubsection{Мобильный телефон -- универсальный беспроводной контроллер}

\begin{enumerate}
\item мэппинг OSC --> MIDI
\item слайдеры и кнопки по OSC
\item управление программами и синтезаторами: Logic, Kontact,
\item объекты \pdobj{spigot}, \pdobj{change}, \pdobj{b}, \pdobj{f}, \pdobj{select}, \pdobj{route} 		
\end{enumerate}

\subsection{Усовершенствуем синтезатор}

\subsubsection{Изменяем форму волны}
\begin{enumerate}
	\item синусоида, пила, меандр
	\item искажение сигнала, дисторшн
	\item объекты \pdobj{expr\~}, \pdobj{clip~}, \pdobj{tanh\~}
\end{enumerate}

\subsubsection{Амплитудная модуляция}

\begin{enumerate}
\item Кольцевая модуляция
\item Работа с источниками звука, воспроизводим файлы, обрабатываем звук с микрофона
\item Создаём голос робота
\item объекты \pdobj{adc\~}, \pdobj{readsf~}
\end{enumerate}
	
\subsubsection{Частотная модуляция}



\subsubsection{Огибающая}
\begin{enumerate}
	\item ADSR
\end{enumerate}

\subsubsection{Создаем звуки ударных}
\begin{enumerate}
	\item объекты \pdobj{lp\~}, \pdobj{hp\~}, \pdobj{bp\~}
	\item Snare, Hat, Kick	
\end{enumerate}

\subsubsection{Создаем drum-machine}
\begin{enumerate}
	\item паттерны, циркадные ритмы
\end{enumerate}

\subsubsection{Аддитивный синтез -- создаем электроорган}
\begin{enumerate}
	\item объекты \pdobj{cheby\~}
\end{enumerate}

\subsubsection{Перформанс на синтезаторах}

\section{Управляющая логика}

\subsection{Идиомы PureData}
\begin{enumerate}
	\item управляющие структуры, списки, циклы
	\item объекты \pdobj{list}, \pdobj{timer}
	\item сохранение настроек синтезатора
\end{enumerate}

\subsubsection{Организация кода проекта. Модули}
\begin{enumerate}
	\item пути поиска, написание собственных библиотек
	\item использование git для контроля версий
	\item объекты \pdobj{\$0}
	\item делаем синтезатор в виде отдельного модуля
\end{enumerate}

\section{Обработка сигналов}
\subsection{Строим микшер}
\begin{enumerate}
	\item объекты \pdobj{pvu\~}, \pdobj{vu\~}
\end{enumerate}

\subsection{Реверберация}
\begin{enumerate}
	\item объекты \pdobj{rev3\~}, \pdobj{freeverb\~}
\end{enumerate}

\hometask{микшер управляемый по OSC}

\subsection{Искажение сигналов}
\begin{enumerate}
	\item объекты \pdobj{clip\~}, \pdobj{tanh\~}
\end{enumerate}

\section{Live-электроника}
\begin{enumerate}
	\item учим компьютер реагировать на звуки исполнителя.
	\item объекты \pdobj{fiddle\~}, \pdobj{bonk\~}
\end{enumerate}

\section{Генеративная музыка}
\begin{enumerate}
	\item История генеративной музыки.
	\item игра Life
	\item L-системы, 
	\item циркадные ритмы.
\end{enumerate}

\section{Запуск патчей Pd на одноплатном компьютере Rasberry Pi}

\section{Программирование графики в программе Processing}

\section{Соединение аудио и графики, создание мультимедиа инсталляции c использованием Arduino и Rasberry Pi}

\section{Выставка работ по "Treatise" Корнелиуса Кардью}

\section{Использование Microsoft Kinect}

\section{OpenFrameworks}

\section{обзор существующих систем для мультимедиа программирования}
\begin{enumerate}
	\item MaxMSP, Jitter, CSound, SuperCollider, Chuck, Fluxus, NodeBox
\end{enumerate}

\end{document}